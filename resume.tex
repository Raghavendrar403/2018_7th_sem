\documentclass{article}

\usepackage{titlesec}
\usepackage{titling}
\usepackage[margin=1.25in]{geometry}

\titleformat{\section}[frame]
{\huge}
{}
{0.1em}
{\filcenter\bfseries\lowercase}

\titleformat{\subsection}
{\bfseries\Large}
{\hspace{-.25in}$\bullet$}
{0.4em}
{}

\titleformat{\subsubsection}[runin]
{\bfseries}
{}
{0em}
{}[:]

\titlespacing{\subsubsection}
{0em}
{0.5em}
{0.5em}

\renewcommand{\maketitle}{
\begin{center} 
{\huge\bfseries
\theauthor}

raghavendrar403@gmail.com --- https://github.com/Raghavendrar403/

\end{center}

}


\begin{document}
\title{R\'esum\'e}
\author{Raghavendr Galeppa}

\maketitle

\section{Technical Skills}

\subsection{Work Flow}

Editor: vim, Debugger: GNU Debugger(gdb), Version Control: git, OS: Arch Linux

\subsection{Languages}

\subsubsection{Programming}

C, C++, Java

\subsubsection{Markup}

{\LaTeX}, HTML, CSS

\subsubsection{Scripting}

Python(basic), shell(basic), R(basic)

\section {Projects}

\subsection{Completed Projects}

\subsubsection{Parallelizing Matrix Multiplication} 

Parallelized Matrix Multiplication using both POSIX Threads api and OpenMP. Compared the results with sequential algorithm 
and then documented the results. Done as a 2 man team. My part was to write the code for Pthreads part and create the documentation
of the results. 
\textbf{https://github.com/Raghavendrar403/multi\_programming}

\subsubsection{Lexical and Syntax Analysis}
Created a very basic Compiler with only Lexical and Syntax Analysis for a subset of Ada/CS programming language. All code written in C.

\textbf{https://github.com/Raghavendrar403/compiler\_design}

\subsubsection{A very basic Neural Network}
Create a very basic Neural Network using numpy package for recognizing hand written digits. It was trained using MNIST dataset and uses Backpropogation Algorithm.

\subsection{Ongoing Projects}
\subsubsection{Semantics Analyzer with using Action symbols}
Trying to build a Semantics Analyzer for the Compiler.

\subsubsection{Decision Tree Visualization for ID3 and CART}
Visualizing Decision Trees for both ID3 and CART Algorithm.

\section{General Skills}
\subsection{Languages}
\subsubsection{Can Speak} 
Kannada, Hindi, Telugu, English

\subsubsection{Can Read}
Kannada, Hindi, English

\section{Education}
\subsection{Class 10}
Kendriya Vidyalaya CGPA: 8.0

\subsection{Class 12}
Kendriya Vidyalaya Aggregate: 75.4

\subsection{B.E Computer Science}
PESIT South Campus Expected Grad August 2019, CGPA: 6.37



\end{document}
